\usepackage{etoolbox}
\usepackage{amsmath, amsthm, amssymb}
\usepackage{a4wide}
\usepackage{setspace}
\onehalfspacing

\RequirePackage[dvipsnames]{xcolor}
\RequirePackage{dsfont}

\usepackage[%
  colorlinks = true,
  citecolor  = RoyalBlue,
  linkcolor  = RoyalBlue,
  urlcolor   = RoyalBlue,
  unicode,
  ]{hyperref}
\usepackage[capitalize]{cleveref}


\usepackage[%
  autocite     = plain,
  backend      = biber,
  doi          = true,
  url          = true,
  giveninits   = true,
  hyperref     = true,
  maxbibnames  = 99,
  maxcitenames = 99,
  sortcites    = true,
  style        = numeric,
  ]{biblatex}
\addbibresource{refs.bib}


\theoremstyle{plain}

\newtheorem{theorem}{Theorem}
\newtheorem{proposition}[theorem]{Proposition}
\newtheorem{lemma}[theorem]{Lemma}

\theoremstyle{definition}
\newtheorem{example}[theorem]{Example}
\newtheorem{definition}[theorem]{Definition}
\newtheorem{remark}[theorem]{Remark}
\newtheorem{convention}[theorem]{Convention}
\newtheorem{problem}[theorem]{Problem}

\newcommand{\liff}{\leftrightarrow}
\newcommand{\iffdef}{\stackrel{\mathrm{def}}{\iff}} 
\newcommand{\To}{\Rightarrow}  
\newcommand{\onto}{\twoheadrightarrow}
\newcommand{\plainRel}{\to}
\newcommand{\rel}[1]{\stackrel{#1}{\plainRel}}
\newcommand{\plainrRel}{\leftarrow}
\newcommand{\rrel}[1]{\stackrel{#1}{\plainrRel}}

\newcommand{\mbb}[1]{\mathbb{#1}}
\newcommand{\mbf}[1]{\mathbf{#1}}
\newcommand{\mr}[1]{\mathrm{#1}}
\newcommand{\mc}[1]{\mathcal{#1}}
\newcommand{\ms}[1]{\mathsf{#1}}

\DeclareMathOperator*{\dom}{\mathrm{dom}}
\DeclareMathOperator*{\im}{\mathrm{im}} 
\newcommand{\bN}{\mathbb{N}}
\newcommand{\cP}{\mathcal{P}}
\newcommand{\op}{{\mathrm{op}}}
\newcommand{\sem}[1]{\llbracket{#1}\rrbracket}
\newcommand{\gen}[1]{\langle #1 \rangle}
\newcommand{\isdef}{\stackrel{\mathrm{def}}{=}}
\newcommand{\Reg}{\mathsf{Reg}}
\DeclareMathOperator*{\Spec}{\mathsf{Spec}}
\DeclareMathOperator*{\Clp}{\mathsf{Clp}}
\newcommand{\id}{\mathsf{id}}
\newcommand{\BoolAlg}{\mbf{BoolAlg}}
\newcommand{\BoolSp}{\mbf{BoolSp}}

\newcommand{\pro}{\mathsf{pro}}
\newcommand{\set}{\mathsf{set}}
\newcommand{\disc}{\mathsf{disc}}
\renewcommand{\c}{\mathsf{c}}

\newcommand{\zer}{0}
\newcommand{\one}{1}

\newcommand{\old}[1]{}
